\documentclass[a4paper,11pt]{article}
\usepackage[MeX]{polski}
\usepackage[utf8]{inputenc}

%opening
\title{\textbf{
\begin{flushleft}
Wydział Matematyki i Informatyki Uniwersytetu Warmińsko-Mazurskiego
\end{flushleft}
}}
\author{Bartosz Nowaczewski}

\begin{document}

\maketitle

\begin{abstract}
\textbf{Wydział Matematyki i Informatyki Uniwersytetu Warmińsko-Mazurskiego }(WMiI) – wydział
Uniwersytetu Warmińsko-Mazurskiego w Olsztynie oferujący studia na dwóch kierunkach:
\begin{itemize}
	\item Informatyka
  \item Matematyka
\end{itemize}
w trybie studiów stacjonarnych i niestacjonarnych. Ponadto oferuje studia podyplomowe.
Wydział zatrudnia 8 profesorów, 14 doktorów habilitowanych, 53 doktorów i 28 magistrów.

\end{abstract}

\begin{abstract}

\textbf{Spis treści}

\begin{itemize}
	\item 1 Misja

\item 2 Opis kierunków[1]

\item 3 Struktura organizacyjna

\item 4 Władze Wydziału

\item 5 Historia Wydziału

\item 6 Nowa siedziba Wydziału

\item 7 Adres

\item 8 Przypisy

\item 9 Linki zewnętrzne
\end{itemize}

\textbf{Misja}

Misją Wydziału jest:
\begin{itemize}
	\item Kształcenie matematyków zdolnych do udziału w rozwijaniu matematyki i jej stosowania w innych
działach wiedzy i w praktyce;

 \item Kształcenie nauczycieli matematyki, nauczycieli matematyki z fizyką a także nauczycieli informatyki;

 \item Kształcenie profesjonalnych informatyków dla potrzeb gospodarki, administracji, szkolnictwa oraz życia
społecznego;

 \item Nauczanie matematyki i jej działów specjalnych jak statystyka matematyczna, ekonometria,
biomatematyka, ekologia matematyczna, metody numeryczne; fizyki a w razie potrzeby i podstaw
informatyki na wszystkich wydziałach UWM.

\end{itemize}

\end{abstract}

\section{}

\end{document}